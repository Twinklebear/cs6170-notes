% This is a LaTex template for scribe notes for 
% CS 6170: Computational Topology. 
% This template is adapted from a template thanks to 
% Jeff A. Bilmes @ U. Washington and Alistair Sinclair @ Berkeley. 

\documentclass{article}
\usepackage{times,amsmath,amsthm,amsfonts,eucal,graphicx}
\usepackage{amssymb}
\usepackage{enumitem}
\usepackage{color}

\setlength{\oddsidemargin}{0.25 in}
\setlength{\evensidemargin}{-0.25 in}
\setlength{\topmargin}{-0.6 in}
\setlength{\textwidth}{6.5 in}
\setlength{\textheight}{8.5 in}
\setlength{\headsep}{0.75 in}
\setlength{\parindent}{0 in}
\setlength{\parskip}{0.1 in}

%
% The following commands set up the lecnum (lecture number)
% counter and make various numbering schemes work relative
% to the lecture number.
%
\newcounter{lecnum}
\renewcommand{\thepage}{\thelecnum-\arabic{page}}
\renewcommand{\thesection}{\thelecnum.\arabic{section}}
\renewcommand{\theequation}{\thelecnum.\arabic{equation}}
\renewcommand{\thefigure}{\thelecnum.\arabic{figure}}
\renewcommand{\thetable}{\thelecnum.\arabic{table}}

%
% A few symbols that we will be using often in this course.
\newcommand{\indep}{{\bot\negthickspace\negthickspace\bot}}
\newcommand{\notindep}{{\not\negthickspace\negthinspace{\bot\negthickspace\negthickspace\bot}}}
\newcommand{\definedtobe}{\stackrel{\Delta}{=}}
\renewcommand{\choose}[2]{{{#1}\atopwithdelims(){#2}}}
\newcommand{\argmax}[1]{{\hbox{$\underset{#1}{\mbox{argmax}}\;$}}}
\newcommand{\argmin}[1]{{\hbox{$\underset{#1}{\mbox{argmin}}\;$}}}

%
% The following macro is used to generate the header.
%
\newcommand{\lecture}[4]{
   \pagestyle{myheadings}
   \thispagestyle{plain}
   \newpage
   \setcounter{lecnum}{#1}
   \setcounter{page}{1}
   \noindent
   \begin{center}
   \framebox{
      \vbox{\vspace{2mm}
    \hbox to 6.58in { {\bf CS 6170 Computational Topology: Topological Data Analysis
                        \hfill University of Utah} }
    \hbox to 6.58in { {\bf Spring 2017
                        \hfill School of Computing} }
       \vspace{4mm}
       \hbox to 6.28in { {\Large \hfill Lecture #1: #2  \hfill} }
       \vspace{2mm}
       \hbox to 6.28in { {\it Lecturer: {\it Prof. Bei Wang {\tt <beiwang@sci.utah.edu>}} \hfill Scribe: #3} }
      \vspace{2mm}}
   }
   \end{center}
   \markboth{Lecture #1: #2}{Lecture #1: #2}
   \vspace*{4mm}
}

%
% Convention for citations is authors' initials followed by the year.
% For example, to cite a paper by Leighton and Maggs you would type
% \cite{LM89}, and to cite a paper by Strassen you would type \cite{S69}.
% (To avoid bibliography problems, for now we redefine the \cite command.)
% Also commands that create a suitable format for the reference list.
\renewcommand{\cite}[1]{[#1]}
\def\beginrefs{\begin{list}%
        {[\arabic{equation}]}{\usecounter{equation}
         \setlength{\leftmargin}{2.0truecm}\setlength{\labelsep}{0.4truecm}%
         \setlength{\labelwidth}{1.6truecm}}}
\def\endrefs{\end{list}}
\def\bibentry#1{\item[\hbox{[#1]}]}

%Use this command for a figure; it puts a figure in wherever you want it.
%usage: \fig{NUMBER}{CAPTION}{.eps FILE TO INCLUDE}{WIDTH-IN-INCHES}
\newcommand{\fig}[4]{
			\begin{center}
	                \includegraphics[width=#4,clip=true]{#3} \\
			Figure \thelecnum.#1:~#2
			\end{center}
	}
% Use these for theorems, lemmas, proofs, etc.
\newtheorem{theorem}{Theorem}[lecnum]
\newtheorem{lemma}[theorem]{Lemma}
\newtheorem{proposition}[theorem]{Proposition}
\newtheorem{claim}[theorem]{Claim}
\newtheorem{corollary}[theorem]{Corollary}
\newtheorem{definition}[theorem]{Definition}
% \newenvironment{proof}{{\bf Proof:}}{\hfill\rule{2mm}{2mm}}

% **** IF YOU WANT TO DEFINE ADDITIONAL MACROS FOR YOURSELF, PUT THEM HERE:

\begin{document}
%FILL IN THE RIGHT INFO.
%\lecture{**LECTURE-NUMBER**}{**DATE**}{**LECTURER**}{**SCRIBE**}
\lecture{4}{April 4, 2017}{Chris Brooks, Will Usher}

% **** YOUR NOTES GO HERE:

Thursday 4/6 Quiz
Given a 1 or 2 dimensional function, give reeb graph, contour tree or mapper

Topics:
Extended persistence
Elevation function
Protein Docking

\section{Extended persistence}

There is something not satisfactory about ordinary persistence. Consider a ``deformed torus'' with a height function $f$. Going from bottom to top, the sublevel sets $\mathbb{X}=f^{-1}(-\infty,a]$ are a bowl at the first local min, two bowls at the next local min, glued bowls at the saddle point, a bowl with a handle at the second saddle point (note the tunnel) and so on.

We can pair the critical points $(b,c)$ on the basis of a component being born at $b$ and killed at $c$. Same for $(f,g)$. We can also pair the global minimum with the global maximum $(a,h)$. This is the first part of ``extended persistence'' and it captures an \emph{essential feature} of the space. We call $(a,h)$ an \emph{extended persistence pair}. Another essential feature would be the tunnel described by the pair of saddle points $(d,e)$.

We can make a Reeb graph for this space. The smallest branches correspond to the ordinary pairs, and the remaining vertices can be associated with the extended persistence pairs.

These features of a topological space can be useful in analyzing the interaction of proteins and other molecules since these interactions are dependent on the compatibility of their shapes. The non-essential pairs from the previous example give a sense of the size of that feature; for instance, the pair $(c,d)$ gives the size of the corresponding protrusion.

We can also do ``persistence simplification''. For example, if we push the point $b$ up past $c$, we remove two critical points and the protrusion associated with it.

This type of anaylsis can be made better by adding a separate height function in another direction. This gives us different critical points and new critical pairs which highlight the shape's features along that direcion.

\section{Elevation function}

An elevation function $E:\mathbb{X}\to\mathbb{R}$ for any point $x\in\mathbb{X}$ gives us $E(x)$, which is the persistence of $x$ when it becomes critical. In the previous examples the critical points appeared where the normal direction aligned with the height direction.

In the first example (the deformed torus), the elevation function is defined by $E(f)=E(g)=|t_g-t_f|$, $E(a)=E(h)=|t_h-t_a|$, etc. If our height function is not Morse, there may be multiple critical points that can be paired with another critical point. This can happen if we sample directions on a shape to do this kind of analysis; a small perturbation can eliminate the ambiguity.

The notion of elevation function is useful in protein analysis since if two proteins' shapes are compatible, then there is a direction in which the elevation value of its features should be similar where the proteins meet.

\end{document}

